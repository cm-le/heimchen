\documentclass{scrartcl}
\usepackage[utf8]{inputenc}
\usepackage[ngerman]{babel}
\begin{document}
\def\DB{Datenbank der Heimatkundlichen Sammlung Seuzach}
\section{Einleitung}

Die neue Datenbank der Heimatkundlichen Sammlung Seuzach 

\section{Grundkonzepte der Datenbank}

Jede Datenbank ist aber nur so gut, wie die Daten, die sie enthält,
deshalb ist es unerläßlich, dass alle, die mit der Erfassung und
Pflege der Daten betraut sind, ein gutes Verständnis der
zugrundeliegenden Konzepte und der sich daraus ergebenden Richtlinien
(``Was wird wie/wo erfaßt?'')  haben.

Als relationale Datenbank verwaltet die \DB{} ganz allgemein Dinge und ihre Beziehungen
(Relationen) zueinander.

Die Dinge, die die Datenbank verwaltet, sind dabei
\begin{itemize}
\item Personen
\item Orte
\item Stichwörter
\item Bilddateien \textit{und}
\item die eigentlichen Sammlungsstücke
\end{itemize}

Die Beziehungen zwischen diesen Dingen sind z\.B, dass bestimmte
Personen auf bestimmten Bildern zu sehen sind, dass bestimmte Stichwörter
zu bestimmten Sammlungsstücken gehören etc.

Die zugrundeliegende Idee ist, dass eine Person -- mit ihrem Geburts- und Sterbedaten,
Spitznamen (``Millimeter''), und wissenswertem zu dieser Person -- nur
einmal in der Datenbank erfaßt wird, und alle mit dieser Person in Verbindung
stehenden Dinge (Bilder, Orte) über entsprechende Verknüpfungen trotzdem direkt
bei dem Personeneintrag angezeigt werden können.

\paragraph{Personen} Bei einer Person ist es relativ klar, wer damit gemeint ist:
Jeder Mensch ist einzigartig: Er wird geboren, er lebt und stirbt, aber immer ist
es eine klar abgegrenzte Person.

\paragraph{Orte} Bei einem Ort ist es schon nicht mehr so deutlich, was damit gemeint ist.
Da ist einerseits die räumliche Unschärfe: Ist das Aussendorf ein Ort, oder doch
eher speziell die Alte Poststrasse, oder ganz konkret der Spezereiladen der Frau Schäppi?
Für manche Bilder oder Suchanfragen würde man eher die eine, für andere eher die andere
Einteilung wählen. 

Dann ist da die zeitliche Unschärfe. Der Migros ist ein jedem Seuzacher bekannter
Ort, der auf vielen Bildern zu sehen ist. Ist es aber der gleiche Ort wie das
Bauernhaus Gassmann, das vorher an dieser Stelle stand? Auch hier kann man je nach
Anwendungsfall zu verschiedenen Schlüssen kommen.

In der \DB{} wurde auf eine hierarchiche Erfassung von Orten
verzichtet, da sich eine gewisse Zusammengehörigkeit aus dem
Straßennamen und der geographischen Entfernung (zu allen Orten werden
die geographischen Koordinaten erfaßt) ergibt. Häuser bei
Besitzerwechsel bleiben der gleiche Ort, auch Straßen bei
Umbenennungen. Ob man verschiedene Gebäude, die zu verschiedenen Zeiten
auf dem gleichen Fleck standen, zu einem Ort zusammenfaßt bleibt sicher
eine Einzelfallentscheidung, die auch mit dem Charakter der entsprechenden
Gebäude zusammenhängt.

\paragraph{Stichwörter} Zu den Sammlungsstücken -- aber auch zu Orten und Personen --
können Stichwörter erfaßt werden. Die explizite Vergabe eines
Stichwörtes zu einem Eintrag unterscheidet sich grundlegend von einem
einfachen Kommentar. Bei dem Stichwort ist Schweibweise und Wortwahl
vorgegeben (zB Wümet, Traubenernte oder Traubenlese).

Dies erleichtert die Suche und bietet -- im Gegensatz zu Kommentaren
-- auch eine einfache Übersicht, was für Stichwörter überhaupt
verfügbar sind, d.\,h.  wonach man überhaupt suchen könnte.

Damit geht aber auch eine Verantwortung einher, möglichst sinnvolle
Stichwörter zu wählen, und die gewählten Stichwörter auch konsequent
bei allen betreffenden Einträgen hinzuzufügen.

Um sich in der Vielzahl der Stichwörter besser zurechtzufinden, ist es
in der \DB{} vorgesehen, dass jedes Stichwort einer ``Kategorie''
zugehörig ist.

\paragraph{Bilddateien} Eine Bilddatei wird beim hochladen automatisch in verschiedene
Auflösungen umgerechnet. Von einem Sammlungsstück, z\.B. einem Glasdia, können verschiedene
Digitalisierungen, und damit verschiedene Bilddateien vorliegen.

Markierungen von Personen auf Bildern müssen sich aber, da sie sich
auf eine konkrete Position innerhalb des Bildes beziehen, auf eine konkrete Bilddatei beziehen.

\paragraph{Sammlungsstücke} Dies sind die eigentlichen Einträge für die konkreten Sammlungsstücke --
Bilder, Gegenstände etc.

\subsection{Was ist ein Sammlungsstück, was ist ein Stichwort, was ist ein Bild?}

Während vor einem Jahrhundert Fotographien mit teuren Gerätschaften
und oft von speziell ausgebildeten Fotographen erstellt wurden, und
eine Fotographie demzufolge eine Seltenheit war, die speziellen
Anlässen (Hochzeit, Konfirmation) oder besonders erinnerungswürdigen
Momenten vorbehalten war, können heutzutage Fotos duzend- und
hundertfach von jedem HANDY-Besitzer ohne besonderen Aufwand erstellt
werden.

Entsprechend ändert sich der Charakter eines Bildes in der Datenbank
mit der Zeit.

Die älteren Bilder stellen zweifellos jedes für sich ein
eigenständiges Sammlungsstück dar, welches sorgfältig kommentiert und
mit entsprechenden Markierungen und Stichwörtern versehen werden
sollte.

Aber sollte man jeden Schnappschuß von der 750-Jahrfeier auf die
gleiche Art behandeln, auch wenn er nur nochmal das Büffet aus einer
weiteren Perspektive zeigt?

Gleichzeitig kann man aber auch bei alten Fotos, z\.B. verschiedenen
Fotos der gleichen Ausflugsgesellschaft, fragen, ob sie nicht besser
zu einem Sammlungsstück zusammengefaßt werden sollten.

Eine gute Richtschnur mag sein, dass man jedem Sammlungsstück einen
Namen geben muß. Wenn mehrere Bilder eigentlich den gleichen Namen
bekommen müßten (``Die Feuerwehr präsentiert sich im Freibad Weiher''),
dann sollte man sie zu einem Sammlungsstück zusammenfassen.

Wenn Bilder verschiedenes zeigen, aber doch unter bestimmten
Gesichtspunkten zusammengehören, sollte ein entsprechendes Stichwort
wergeben werden.

\subsection{Kommentare, Kommentare}

Die Anekdoten zu Personen, Begebenheiten oder Häusern bilden -- neben
den historischen Fotographien -- den eigentlichen Schatz der \DB{}.

Durch die expliziten Verknüpfungen zwischen den Einträgen ist es
in der neuen Datenbank nicht mehr nötig, diese Anekdoten oder Hinweise
bei jedem Bild zu wiederholen. Statt dessen können interessante
Dinge zur Person bei der jeweiligen Person erfaßt werden, oder
Dinge die einen Ort betreffen, ebendort.

Eine Bemerkung wie ``Ich wollte unbedingt das neue, modische, weisse
Brautkleid, sonst hätte ich nicht geheiratet.'' bezieht sich hingegen
eher auf das entsprechende Foto.

Andere Kommentare wie ``Hans und Seline Keller-Schwarz zogen wegen
Arbeitsmangel 1920 nach Belfort'', können durchaus bei beiden
betroffenen Personen stehen. Da die Personen aber ebenfalls explizit
verknüpft sind (=sein sollten), wäre der Kommentar auch bei Seline
sichtbar, auch wenn er nur bei Hans gegeben ist.

Schließlich gibt es noch rein technische Kommentare
(``Original-Glasdia ist qualitativ besser''), die sich auf die
konkrete Bilddatei beziehen. 



\section{Funktionen der Datenbank}
\subsection{Neue Einträge erstellen}

Ein neuer Eintrag -- ein neues Sammlungsstück -- wird nach dem Login und
Auswahl des Menüpunktes ``Sammlung'' durch den Button \fbox{Neuer~Eintrag}
oben rechts erstellt.

Viele der Felder -- z\.B. Inventar-Nr. -- können meist leer bleiben,
es empfielt sich jedoch sehr, sich eine treffende Bezeichnung zu überlegen.

Auch ist die Abgrenzung zwischen ``Bezeichnung'' und ``Kommentar''
nicht immer einfach.

Zusätzlich sind viele Dinge wie ``Foto s/w'' oder das Datum der Aufnahme,
z.\.T. auch die abgebildeten Orte und Personen besser durch explizite
Verknüpfungen bzw. Stichworte erfaßt als alles in die Bezeichnung zu quetschen.

Andererseits bleibt -- wenn man Personen und Orte wegnimmt -- nicht viel übrig,
so dass eine gewisse Redundanz (Nennung des Ortes in der Bezeichnung, als auch
explizite Verknüpfung des Ortes) wohl sinnvoll bleibt.


\subsection{Bilddateien und Bildmarkierungen hinzufügen}

Nach Erstellen des Eintrages (z.\.B. ``Vereinsreise der Seuzemer Turner'')
ist es meist sinnvoll, eine oder mehrere Bilder hochzuladen. Derzeit werden die Bildformate
JPEG und PNG unterstützt, andere sind einfach möglich.

Es besteht die Möglichkeit, zu einem Bild einen Anhang zu erstellen, auf diese Weise
ist es z\.B. möglich, ein Poster als Bild sowie als PDF (oder als Word-Datei etc)
abzulegen. Tatsächlich kann diese Funktionalität genutzt werden, um ein simples Dokumentenarchiv
der Heimatkundlichen Sammlung zu realisieren.




\subsection{Suche}

\section{Datenbereinigung}

\section{Ausblick}

\appendix

\section{Technische Details}

\end{document}
